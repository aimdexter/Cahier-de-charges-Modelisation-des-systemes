\section{Moyens technologiques requis}
\subsection{Modélisation :}
\setlength{\parindent}{5ex}
\subsubsection{Objectifs}

Dans le cadre de notre projet informatique pour le SI intitulé \textbf{<Gestion de concours>}, le recours à la modélisation procure de nombreux
avantages qui agissent sur:

\begin{itemize}
	\item[-] \textbf{La modularité :} Servir aussi bien un candidat unique que des centaines ou des milliers de candidats simultanés.
	\item[]
	\item[-] \textbf{L’abstraction  :} Réduire la complexité et obtenir une conception et une implémentation plus efficaces.
	\item[]
	\item[-] \textbf{La dissimulation :} Garder une emprunt d’identité ou identité du client.
	\item[]
	\item[-] \textbf{La structuration cohérente des fonctionnalités et des données :} Une architecture scalable.
\end{itemize}

\setlength{\parindent}{20pt}
\setlength{\parskip}{1em}

\subsubsection{Systems Modeling Language (SysML)}

Le langage SysML, commun à tous les champs disciplinaires, est composé de diagrammes qui permettent d’aborder
plus facilement les systèmes pluritechniques, que ce soit en phase de conception ou en phase d’analyse d’un existant.\par

\textbf{Le langage SysML nous offres :}
\begin{enumerate}
	\item diagramme des exigences (requirement diagram)
	\item diagramme des cas d’utilisation (use case diagram)
	\item diagramme de séquence (sequence diagram)
	\item diagramme d’état (state diagram)
	\item diagramme de définition de blocs (definition block diagram)
	\item diagramme de blocs internes (internal block diagram)
\end{enumerate}
\vfill
\noindent\makebox[\linewidth]{\rule{.8\paperwidth}{.6pt}}\\[0.2cm]
ENSMR - Modélisation des systèmes - 2022/2023 \hfill Gestion des concours
\noindent\makebox[\linewidth]{\rule{.8\paperwidth}{.6pt}}
\pagebreak

\subsection{Infrastructure cloud}
\begin{itemize}
	\item[-] \textbf{Amazon Web Services (AWS)} : permet un gain d’agilité. Le positionnement « à la demande » assure ainsi de toujours n’utiliser
		que ce qui est nécessaire. Héberger ses applications dans le cloud AWS est l’une des utilisations les plus pertinentes.
\end{itemize}

\subsection{Base de données}
\begin{itemize}
	\item[-] \textbf{PostgreSQL} : Ce système de gestion de base de données permet de gérer nombreux types des datas avec ses fonctionnalités basiques.
		Les données les plus complexes peuvent très bien être traitées pour une optimisation de la performance de l’entreprise.
\end{itemize}

\subsection{Côté client (Front-end)}
\begin{itemize}
	\item[-] \textbf{Next.js} : Next.js étend la bibliothèque React originale de Facebook et le package create-react-app pour fournir un cadre React 
    extensible, facile à utiliser et à l’épreuve de la production.
\end{itemize}

\subsection{Côté serveur(Back-end)}
\begin{itemize}
	\item[-] \textbf{Django} : Django est sorti en 2005 et est devenu l’un des frameworks Web incontournables pour les développeurs Web. Il a été créé en 
    tant que framework sur le langage de programmation Python. Avec un ensemble de fonctionnalités appropriées, Django réduit la quantité de code 
    trivial qui simplifie la création d’applications Web et se traduit par un développement plus rapide.
\end{itemize}

\vfill
\noindent\makebox[\linewidth]{\rule{.8\paperwidth}{.6pt}}\\[0.2cm]
ENSMR - Modélisation des systèmes - 2022/2023 \hfill Gestion des concours
\noindent\makebox[\linewidth]{\rule{.8\paperwidth}{.6pt}}
\newpage