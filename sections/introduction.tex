\section{Introduction}
%------------------------------------------------------------------------------------------
\setlength{\parindent}{5ex}
Le cahier des charges constitue le cœur de la phase d’analyse d‘un projet et de la détermination 
des besoins. Il permet d’assurer la communication entre les acteurs du projet. Cette étape 
s’avère cruciale dans tout projet, elle concerne surtout la mise en route de ses aspects. 
En outre, le cahier des charges permet de définir le contexte de fonctionnement, ainsi que 
les objectifs que doit atteindre la solution.


Dans le cadre de ce présent projet, nous nous sommes penchés à mettre en pratique les 
différentes connaissances acquises en cours de modélisation des systèmes dans la réalisation 
d’un projet concernant la gestion des concours. Et ce à commencer par ce cahier des charges
 qui présente la plateforme allant de l’origine et de la nature du projet jusqu’aux différentes 
 exigences techniques à tenir en compte.
%------------------------------------------------------------------------------------------
\setlength{\parindent}{5ex}
\vfill
\noindent\makebox[\linewidth]{\rule{.8\paperwidth}{.6pt}}\\[0.2cm]
ENSMR - Modélisation des systèmes - 2022/2023 \hfill Gestion des concours
\noindent\makebox[\linewidth]{\rule{.8\paperwidth}{.6pt}}
\newpage