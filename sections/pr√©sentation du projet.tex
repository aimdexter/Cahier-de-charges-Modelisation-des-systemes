\section{Présentation du projet}
%------------------------------------------------------------------------------------------
\subsection{Contexte et problématique}
%------------------------------------------------------------------------------------------
\setlength{\parindent}{5ex}
Considérant qu’un concours est un ensemble d'épreuves mettant en compétition des 
candidats afin d’accéder à des emplois, des écoles, des récompenses ou des avantages 
publiques. Cet examen où ne sont admis qu'un nombre limité et déterminé à l'avance de 
candidats, constitue une phase d’une importance cruciale dans la carriére 
professionnelle de tout individu. Et dû à son péremptoire et sa grandeur, 
l’organisation et la gestion du concours demeurent une perplexité pour tous 
les organisateurs. Les concours (et plus généralement la sélection) sont critiqués 
parce qu'ils reproduisent les inégalités sociales, voire les renforcent, mais dans 
tout critère d'évaluation qu'il s'agisse de la culture générale, de l'histoire, des 
langues ou même des maths, l’égalité entre les candidats est primordiale.
\newline\indent

Il n'existe pas de remède miracle ni d'évaluation ojectivée à 100\% et, à défaut de 
meilleure solution, la contrariété de gerer un concours existe toujours et engendrent 
des réactions negatives, encore problématiques pour les moins favorisés.
En effet tout projet est avant tout un choix pour l'avenir et constitue à ce titre un 
investissement, a ce fait le projet de gestion de concours aura comme primauté 
établir une démarche claire et précise permettant à tout candidat de postuler sa 
candidature, ainsi que le suivi de son dossier dans toutes ses étapes.


%------------------------------------------------------------------------------------------
\subsection{Objectifs et périmètre}
%------------------------------------------------------------------------------------------
\setlength{\parindent}{5ex}
\setlength{\parindent}{5ex}
Notre projet consiste à la réalisation d'une application web visant la gestion des 
modalités du concours. Il est destiné, dans un premier temps, à un établissement 
universitaire donné organisant un concours annuel qui permet l'accès des étudiants 
issus du baccalauréat marocain ou titulaires des autres diplômes tels que le DEUG, 
DUT ou master selon les besoins et les spécifications de l'offre du concours. Cette 
plateforme facilite la procédure pour passer le concours au sein de l'école depuis la 
présélection jusqu'à l'inscription officielle. 

Dans un 2ème temps, notre projet est dédié aux étudiants leur permettant de postuler 
pour le concours à distance  et de suivre toutes les nouveautés relatives aux 
procédures, et d'avoir un contact direct avec les responsables du concours en cas 
d'un problème technique, ce qui fluidifie le processus et le rend beaucoup plus 
flexible en terme du temps.\vfill
\noindent\makebox[\linewidth]{\rule{.8\paperwidth}{.6pt}}\\[0.2cm]
ENSMR - Modélisation des systèmes - 2022/2023 \hfill Gestion des concours
\noindent\makebox[\linewidth]{\rule{.8\paperwidth}{.6pt}}
%------------------------------------------------------------------------------------------
\newpage