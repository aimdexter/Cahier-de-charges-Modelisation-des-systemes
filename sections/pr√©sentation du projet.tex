\section{Présentation du projet}
%------------------------------------------------------------------------------------------
\subsection{Contexte et problématique}
%------------------------------------------------------------------------------------------
\setlength{\parindent}{5ex}
Nous Nous Nous Nous Nous Nous Nous Nous Nous Nous Nous Nous Nous Nous Nous Nous Nous Nous Nous Nous Nous Nous
Nous Nous Nous Nous Nous Nous Nous Nous Nous Nous Nous Nous Nous Nous Nous Nous Nous Nous Nous Nous Nous Nous
\newline\indent

En effet, ce projet nous apportera une En effet, ce projet nous apportera une En effet, ce projet nous apportera une En effet, ce projet nous apportera une
En effet, ce projet nous apportera une En effet, ce projet nous apportera une En effet, ce projet nous apportera une En effet, ce projet nous apportera une
\newline\indent

Ce projet nous permettra de mettre en pratique Ce projet nous permettra de mettre en pratique Ce projet nous permettra de mettre en pratique
Ce projet nous permettra de mettre en pratique Ce projet nous permettra de mettre en pratique Ce projet nous permettra de mettre en pratique

%------------------------------------------------------------------------------------------
\subsection{Objectifs}
%------------------------------------------------------------------------------------------
\setlength{\parindent}{5ex}
\begin{enumerate}
  \item Another point I want to make
        \begin{enumerate}
          \item Another point I want to make
          \item Another point I want to make
        \end{enumerate}
  \item Another point I want to make
        \begin{enumerate}
          \item Another point I want to make
          \item Another point I want to make
        \end{enumerate}
  \item Another point I want to make
        \begin{enumerate}
          \item Another point I want to make
          \item Another point I want to make
        \end{enumerate}
  \item Another point I want to make
        \begin{enumerate}
          \item Another point I want to make
          \item Another point I want to make
        \end{enumerate}
  \item Another point I want to make
        \begin{enumerate}
          \item Another point I want to make
          \item Another point I want to make
        \end{enumerate}
  \item Another point I want to make
        \begin{enumerate}
          \item Another point I want to make
          \item Another point I want to make
        \end{enumerate}
\end{enumerate}\vfill

\noindent\makebox[\linewidth]{\rule{.8\paperwidth}{.6pt}}\\[0.2cm]
ENSMR - Modélisation des systèmes - 2022/2023 \hfill Gestion des concours
\noindent\makebox[\linewidth]{\rule{.8\paperwidth}{.6pt}}
\newpage

%------------------------------------------------------------------------------------------
\subsection{Périmètre}
%------------------------------------------------------------------------------------------
\setlength{\parindent}{5ex}
Le premier Le premierLe premierLe premierLe premierLe premierLe premierLe premierLe premierLe premierLe premierLe premierLe.

Pour réaliser le cahier des charges ou les différents rapports de soutenances, l'éditeur \LaTeX{} en ligne Overleaf sera employé.

Enfin, pour travailler en groupe, le logiciel de contrôle de version GitHub sera utilisé, permettant de suivre à la trace chaque évolution du projet avec les modification de chaque étudiants du groupe.

\vfill
\noindent\makebox[\linewidth]{\rule{.8\paperwidth}{.6pt}}\\[0.2cm]
ENSMR - Modélisation des systèmes - 2022/2023 \hfill Gestion des concours
\noindent\makebox[\linewidth]{\rule{.8\paperwidth}{.6pt}}
%------------------------------------------------------------------------------------------
\newpage